\section{ABNF Differences from RFC-5234}
\section{XDR Differences from RFC-4506}

\subsection{RPCGEN++ has support for ABNF bit fields}
The ability to assign the individual bits in objects are useful
when you have boolean or other values that do not need to have
their own 32-bit storage.
Or they could represent the bit values needed to interface
with hardware.

They are declared by appending a colon (:) and a decimal number
after the or rule name or terminal value. Example:

\begin{figure}
\begin{verbatim}

RuleWith32Bits:32 =  %x0:8 Value1:4 Value2:4 uint16_t:16

\end{verbatim}
\caption{A 32-bit ABNF example}
\label{fig:A32BitExampleABNF}
\end{figure}

The number of bits on the left side of the equal sign (=)
must equal to the total number of bits on the right side.
Pad with values such as \%0:8, \%0:2, \%0:12, or whatever number
of bits you need to pad in order to make the number of bits
on each side of the equal sign the same.

You must declare them with the least significant bits on
the right most of the equal sign.
In figure \ref{fig:A32BitExampleABNF}, the 16 bit
value would be in bit positions 0-15, Value2 would
be in bit positions 16-19, and Value2 would be in bit
positions 20-23, the upper 8 bits (24-32) are 8-bits
of padding to make the total 32-bits.

\textit{For signed values}, make sure that the value you pass
into the encoding call is correctly signed for the bit
width you assigned in the ABNF file.
If you specify a 4-bit value as signed
and pass it in as an 8-bit signed value, the sign bit will not
be encoded.

The \textit{bitfield} values can be signed or unsigned and
are XDR encoded and XDR decoded on their own.
Once encoded the \textit{bitfield} itself is sent as
opaque data.

In this example, the \textit{bitfield} is 64-bits and
contains multiple values that are not byte aligned:
\begin{figure}
\begin{verbatim}

             ; With:
             ; DeviceEnable as 'A' in the bit diagram.
             ; DeviceMode as 'B' in the bit diagram.
             ; DeviceRotation as 'C' in the bit diagram.
             ; DeviceStatus as 'D' in the bit diagram.
             ; DevicePosition as 'E' in the bit diagram.
             ; And %0:24 is zero padding to fill the
             ; 64-bit value.
             ;
             ; With ABNF, the most significat bits are
             ; on the left and the least significat
             ; values are to the right.
             ;
ALargeOne:64 = %0:21 DevicePosition:32 DeviceStatus:4 DeviceRotation:4
               DeviveMode:2 DeviceEnabled:1
\end{verbatim}
\caption{Packed 64-bit device ABNF example}
\label{fig:ALargeOneABNF}
\end{figure}
Unlike the XDR bitfields, no automatic padding is done.
You will need to specify any padding in the ABNF file.
XDR and this implementation sends data in chunks of 32-bits.
The 'A', 'B', 'C', 'D', and 'E' comments in figure
\ref{fig:ALargeOneABNF} are represented as the bits in
the 64-bit figure \ref{fig:ALargeOneBitsABNF}.
\begin{itemize}
\item 'A' is one bit wide.
\item 'B' is two bits wide.
\item 'C' is four bits wide.
\item 'D' is four bits wide.
\item 'E' is 32 bits wide.
\item '0' is 21 padding bits.
\end{itemize}
\begin{figure}
  \includesvg{64bitBitField.svg}
  \caption{64-bit 'ALargeOne' bitfield}
  \label{fig:ALargeOneBitsABNF}
\end{figure}

RPCGEN++ does not impose a size limit on the bit width.

\begin{figure}
\begin{verbatim}
A192BitOne:192 =   ... define up to 192 bits ...
\end{verbatim}
\caption{Packed 192-bit device example}
\label{fig:A192BitABNF}
\end{figure}

