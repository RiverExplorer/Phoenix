\section{Building RPCGEN++}
The entire Phoenix build system is designed to build everything
on Linux for all platforms.
It uses the GNU compilers on Linux.
The mingw compilers to cross compile for Windows.
And the android NDK package to cross compile for Android.
The build has been tested on the Android based Oculus virtual
reality headsets.

The build uses the POSIX make command.
And can build debug, release, and test coverage libraries,
commands, and test code from the make system.

You can build any of the options below, or '\textbf{make all}
to build everything.

For each build, it builds 2 libraries and commands.
First using dynamic shared libraries, or crates them.
And the second pass builds with a minimum number
of dynamic libraries, using as many static libraries
as possible.

The BuildTools directory contains the Makefiles
that are included to do all of the different builds
and cross compiling.

You will need to modify the paths to the mingw and
android NDK paths to match your installation in the
Makefile.top and Makefile.top.lib files.

You will have to set the ANDROID\_CC, ANDROID\_CXX, ANDROID\_STRIP,
W64\_CC, W64\_CXX, and W64\_STRIP settings.

On Linux the mingw packages are often loaded using the operating
systems installation tools (dnf, apt, yum).

The Android NDK packages have to be downloaded using the
latest google instructions, which seem to change over time.
I used the NDK r26 version.

\subsection{Building For Linux}
From the top of the GIT repository you downloaded,
type one of these commands to build for Linux:

\begin{itemize}
\item '\textbf{make LinuxDebug}'
  To build the Linux libraries and commands with symbol
  and debugging information.
  
  The output will be in BuildTemp/Linux/Debug.
  The static library will be in BuildTemp/Linux/Debug/lib
  and the shared library will be named lib{name}.so.{version}.
  The static library will be named lib{name}.a.
  
  Any commands built will be in BuildTemp/Linux/Debug/bin
  for dynamic.

  And BuildTemp/Linux/Debug/sbin for static programs.

\item '\textbf{make LinuxRelease}'
  To build the Linux libraries and commands with the libraries
  and programs compiled with the optimize switch turned on.
  The necessary symbols will be stripped from them to make
  them smaller.

  The output will be in BuildTemp/Linux/Release.
  The static library will be in BuildTemp/Linux/Release/lib
  and the shared library will be named lib{name}.so.{version}.
  The static library will be named lib{name}.a.
  
  Any commands built will be in BuildTemp/Linux/Release/bin
  for dynamic.

  And BuildTemp/Linux/Release/sbin for static programs.

\item '\textbf{make LinuxTcov}'
  To build the Linux libraries and commands with the libraries
  and programs compiled with the test code and test metric
  collection information built into them.
  See test coverage \ref{TestCoverage} for more information.
  Debug symbols are included in TCOV builds.

  The output will be in BuildTemp/Linux/Tcov.
  The static library will be in BuildTemp/Linux/Tcov/lib
  and the shared library will be named lib{name}.so.{version}.
  The static library will be named lib{name}.a.
  
  Any commands built will be in BuildTemp/Linux/Tcov/bin
  for dynamic.

  And BuildTemp/Linux/Tcov/sbin for static programs.

  \textbf{When you run a TCOV build file or library, it generates a LOT of  output files}. Be sure to read \ref{TestCoverage}.
  
\item '\textbf{make Linux}'
  To build LinuxDebug, LinuxRelease, and LinuxTcov.
\end{itemize}

\subsection{Building For Windows on Linux}
  
From the top of the GIT repository you downloaded,
type one of these commands to build for Windows, 64-bit:

\begin{itemize}
\item '\textbf{make W64Debug}'
  To build the Windows 64-bit libraries and commands with symbol
  and debugging information.

  The output will be in BuildTemp/W64/Debug.
  The static library will be in BuildTemp/W64/Debug/lib
  and the shared library will be named lib{name}.so.{version}.
  The static library will be named lib{name}.a.
  
  Any commands built will be in BuildTemp/W65/Debug/bin,
  for dynamic.
  
  And BuildTemp/W64/Debug/sbin for static programs.

\item '\textbf{make W64Release}'
  To build the Windows 64-bit libraries and commands with the libraries
  and programs compiled with the optimize switch turned on.
  The necessary symbols will be stripped from them to make
  them smaller.

  The output will be in BuildTemp/W64/Release.
  The static library will be in BuildTemp/W64/Release/lib
  and the shared library will be named lib{name}.so.{version}.
  The static library will be named lib{name}.a.
  
  Any commands built will be in BuildTemp/W65/Release/bin,
  for dynamic.
  
  And BuildTemp/W64/Release/sbin for static programs.
  
\item '\textbf{make W64Tcov}'
  To build the Windows libraries and commands with the libraries
  and programs compiled with the test code and test metric
  collection information built into them.
  See test coverage \ref{TestCoverage} for more information.
  Debug symbols are included in TCOV builds.

  The output will be in BuildTemp/W64/Tcov.
  The static library will be in BuildTemp/W64/Tcov/lib
  and the shared library will be named lib{name}.so.{version}.
  The static library will be named lib{name}.a.
  
  Any commands built will be in BuildTemp/W64/Tcov/bin
  for dynamic.

  And BuildTemp/W64/Tcov/sbin for static programs.

  \textbf{When you run a TCOV build file or library, it generates a LOT of  output files}. Be sure to read \ref{TestCoverage}.
  
  You have to run the built programs on a Widows device
  with the matching GNU tools.
  
\item '\textbf{make W64}'
  To build W64Debug, W64Release, and W64Tcov.
\end{itemize}

\subsection{Building For Android on Linux}
  
From the top of the GIT repository you downloaded,
type one of these commands to build for Android, 64-bit:
You will need the GCC NDK build tools in your path.

\begin{itemize}
\item '\textbf{make AndroidDebug}'
  To build the Android 64-bit libraries and commands with symbol
  and debugging information.

  The output will be in BuildTemp/Android/Debug.
  The static library will be in BuildTemp/Android/Debug/lib
  and the shared library will be named lib{name}.so.{version}.
  The static library will be named lib{name}.a.
  
  Any commands built will be in BuildTemp/Android/Debug/bin,
  for dynamic.
  
  And BuildTemp/Android/Debug/sbin for static programs.

\item '\textbf{make W64Release}'
  To build the Android 64-bit libraries and commands with the libraries
  and programs compiled with the optimize switch turned on.
  The necessary symbols will be stripped from them to make
  them smaller.

  The output will be in BuildTemp/Android/Release.
  The static library will be in BuildTemp/W64/Release/lib
  and the shared library will be named lib{name}.so.{version}.
  The static library will be named lib{name}.a.
  
  Any commands built will be in BuildTemp/Android/Release/bin,
  for dynamic.
  
  And BuildTemp/Android/Release/sbin for static programs.
  
\item '\textbf{make W64Tcov}'
  To build the Windows libraries and commands with the libraries
  and programs compiled with the test code and test metric
  collection information built into them.
  See test coverage \ref{TestCoverage} for more information.
  Debug symbols are included in TCOV builds.

  The output will be in BuildTemp/Android/Tcov.
  The static library will be in BuildTemp/Android/Tcov/lib
  and the shared library will be named lib{name}.so.{version}.
  The static library will be named lib{name}.a.
  
  Any commands built will be in BuildTemp/Android/Tcov/bin
  for dynamic.

  And BuildTemp/Android/Tcov/sbin for static programs.

  \textbf{When you run a TCOV build file or library, it generates a LOT of  output files}. Be sure to read \ref{TestCoverage}.

  You have to run the built programs on a Android device
  with the matching GNU tools.
  
\item '\textbf{make Android}'
  To build AndroidDebug, AndroidRelease, and AndroidTcov.
\end{itemize}


\subsection{Filing Bugs Reports}
File bug, issues, and questions on github, at:
\href {https://github.com/RiverExplorer/Phoenix/issues}{Bug and Issue Reports}
\subsection{Test Coverage}
