\section{Protocol Directives}\label{sec:Directives}
Directives control what kind of code the protocol generator creates.

\begin{itemize}
  \item Some directives can be set at the global level only.
  \item Some directives can be set at the global level, and overridden
    for user specified data sets. Or used with the data set and
    not set globally.
  \item Some directives can not be used at the global level and
    can only be used with data sets.
\end{itemize}

\subsection{BITSTREAM Directive}\label{sec:dBitstream}
The [BITSTREAM] directive tells the generator to send
the data out in bit streams.

See \nameref{sec:Bitstreams} (\ref{sec:Bitstreams}, Page \pageref{sec:Bitstreams}).

The [BITSTREM] directive can be used at the global level.
When used at the global lever, it applies to the entire protocol.

When [BITSTREM] is not set at the global level,
and is applied to a scope, data object, or variable,
then it only applies to that scope, data object, or variable.

\subsection{C++ Directive}\label{sec:dCPP}
The [C++] directive instructs the generator to
produce C++ code for what follows.

The [NoC++] directive instructor the generator to not
produce C++ code for what follows.

Both of the [C++] and [NoC++] directives can be set globally,
or per data object.

\subsection{C\# Directive}\label{sec:dCSharp}
The [C\#] directive instructs the generator to
produce C\# code for what follows.

The [NoC\#] directive instructor the generator to not
produce C\# code for what follows.

Both of the [C\# and [NoC\#] directives can be set globally,
or per data object.

\subsection{Default Pack Size}\label{sec:dDefaultPackSize}
When using [BITSTREM] the [DefaultPackSize=n] directive
sets the number of bits considered as part of one word.
When [BITSTREAM=0] is used, then all bits are packed as
if there is no word boundary.

Example:
If hardware takes 4-bit words, you could set [DefaultPackSize=4]
and then:

   int:1 Var1;
   int:1 Var2;

would be packed into 1 4-bit value.

If [PackLR] was also used, and both Var1 and Var2 was set to '1',
then bits would be:

   1100

where the left most '1' is Var1's value
and the right most '1' is Var2's value.

If [PackLR] was also used, and both Var1 and Var2 was set to '1',
then bits would be:

   0011

where the left most '1' is Var1's value
and the right most '1' is Var2's value.

\nameref{sec:dPackSize} (\ref{sec:dPackSize}, Page \pageref{sec:dPackSize})

\subsection{Document Directive}\label{sec:dDocument}

\subsection{Generate CBOR Directive}\label{sec:dCBOR}
The [CBOR] directive informs the generator that the
data set, or data sets that follow are to be transmitted
using the CBOR data encoding and data decoding format.

The [NoCBOR] directives informs the generator that
the data set, or data sets that follow are not to be
transmitted using the CBOR data encoding and data decoding format.

\subsection{Include Driective}\label{sec:dInclude}
The [Include] directive includes other files at the point
where the [Include] directive is located.

Usage example: [Include=FileName.p]

\subsection{Generate JSON Directive}\label{sec:dJSON}
The [JSON] directive informs the generator that the
data set, or data sets that follow are to be transmitted
using the JSON data encoding and data decoding format.

The [NoJSON] directives informs the generator that
the data set, or data sets that follow are not to be
transmitted using the JSON data encoding and data decoding format.

\subsection{Generate NATIVE Directive}\label{sec:dNATIVE}
The [NATIVE] directive informs the generator that the
data set, or data sets that follow are to be transmitted
using the no encoding at all.
The data is sent in host byte order.

The [NoNATIVE] directives informs the generator that
the data set, or data sets that follow are not to be
transmitted using the host bye order.

\subsection{Pack Size Directive}\label{sec:dPackSize}
The [PackSize=n] tells the generator to pack with
a word size of 'n'.

[PackSize] only applies to the object it is applied to.

\nameref{sec:dDefaultPackSize} (\ref{sec:dDefaultPackSize}, Page \pageref{sec:dDefaultPackSize})

\subsection{Generate REST Directive}\label{sec:dREST}
The [REST] directive informs the generator that the
data set, or data sets that follow are to be transmitted
using the REST data encoding and data decoding format.

The [NoREST] directives informs the generator that
the data set, or data sets that follow are not to be
transmitted using the REST data encoding and data decoding format.

\subsection{Namespace Directive}\label{sec:dNamespace}
A namespace is a way of create a new scope.
Names and IDs in a namespace are separate from the same
names and IDs in global, and other namespaces.

[Namespace=games.RiverExplorer] would specify
that all names and IDs are unique and separate
from similar names in other name spaces.

\subsection{PackNBO Directive}\label{sec:dPackNBO}
The [PackNBO] directive tells the generator to generate
code in Network Byte Order over the wire.

This directive applies to packed data.

\subsection{PackLR Directive}\label{sec:dPackLR}
This directive applies to packed data.

If set at the global level, it it the default packing.
If set on an object, then it only applie to the object.

\subsection{PackRL Directive}\label{sec:dPackRL}
This directive applies to packed data.

If set at the global level, it it the default packing.
If set on an object, then it only applie to the object.

\subsection{Public Directive}\label{sec:dPublic}
The [Public] ... todo ...

Also see:
\begin{itemize}
  \item \nameref{sec:dProtected} (\ref{sec:dProtected}, Page \pageref{sec:dProtected})
  \item \nameref{sec:dInternal} (\ref{sec:dInternal}, Page \pageref{sec:dInternal})
  \item \nameref{sec:dPrivate} (\ref{sec:dPrivate}, Page \pageref{sec:dPrivate})
\end{itemize}

\subsection{Protected Directive}\label{sec:dProtected}
The [Protected] ... todo ...

Also see:
\begin{itemize}
  \item \nameref{sec:dPublic} (\ref{sec:dPublic}, Page \pageref{sec:dPublic}),
  \item \nameref{sec:dInternal} (\ref{sec:dInternal}, Page \pageref{sec:dInternal})
  \item \nameref{sec:dPrivate} (\ref{sec:dPrivate}, Page \pageref{sec:dPrivate})
\end{itemize}

\subsection{Internal Directive}\label{sec:dInternal}
The [Internal] ... todo ...

Also see:
\begin{itemize}
  \item \nameref{sec:dPublic} (\ref{sec:dPublic}, Page \pageref{sec:dPublic})
  \item \nameref{sec:dProtected} (\ref{sec:dProtected}, Page \pageref{sec:dProtected})
  \item \nameref{sec:dPrivate} (\ref{sec:dPrivate}, Page \pageref{sec:dPrivate})
\end{itemize}

\subsection{Private Directive}\label{sec:dPrivate}
The [Private] ...todo ...

Also see:
\begin{itemize}
  \item \nameref{sec:sec:dPublic} (\ref{sec:dPublic}, Page \pageref{sec:dPublic})
  \item \nameref{sec:sec:dProtected} (\ref{sec:dProtected}, Page \pageref{sec:dProtected})
  \item \nameref{sec:sec:dInternal} (\ref{sec:dInternal}, Page \pageref{sec:dInternal})
\end{itemize}

\subsection{Generate XDR Directive}\label{sec:dXDR}
The [XDR] directive informs the generator that the
data set, or data sets that follow are to be transmitted
using the XDR data encoding and data decoding format.

The [XDR] directives informs the generator that
the data set, or data sets that follow are not to be
transmitted using the XDR data encoding and data decoding format.

\subsection{Generate XML Directive}\label{sec:dXML}
The [XML] directive informs the generator that the
data set, or data sets that follow are to be transmitted
using the [XML] data encoding and data decoding format.

The [NoJSON] directives informs the generator that
the data set, or data sets that follow are not to be
transmitted using the [XML] data encoding and data decoding format.
