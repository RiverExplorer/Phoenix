\section{XDR Differences from RFC-4506}
\subsection{Place 'struct' Outside of 'union'}
Currently RPCGEN++ does not allow structs to be defined in unions.
Define them out side of the union, and then use them.

For example, this is valid as defined in RFC-4506:
\begin{figure}
\begin{verbatim}
union MasterOfAllTrades (uint32_t Control) {

case 0:
 struct {
   string item<>;
   stringlist Next;
 } element;

 case 1:
   .
   .
   .
\end{verbatim}
\caption{Structure Inside of Union}
\label{fig:StructInFigure}
\end{figure}


Currently, with RPCGEN++, you will need to define your
structure outside of the union as shown in this example:
\begin{figure}
\begin{verbatim}
 struct {
   string item<>;
   stringlist Next;
 } elementObject;

union MasterOfAllTrades (uint32_t Control) {

case 0:
 elementObject element;

 case 1:
   .
   .
   .
\end{verbatim}
\caption{Structure Outside of Union}
\label{fig:StructOutsideFigure}
\end{figure}


\subsection{RPCGEN++ has support for bit XDR fields}
The \textit{bitfield} objects are useful when you have boolean or other
values that do not need to have their own 32-bit storage.
Or they could represent the bit values needed to interface
with hardware.

They are declared by appending a colon (:) and a decimal number
after the data type or variable name.
They are defined using the \textit{bitfield} data type.
In this example, \textit{TwoFourBitValues} is declared
as 8-bits in size, and it contains two 4-bit values.
\begin{figure}
\begin{verbatim}
bitfield TwoFourBitValues:8 {
   int8_t:4 MyVariableName:4;
   int8_t:4 TheOtherFourBits:4;
};
\end{verbatim}
\caption{Two Four Bit values in one 8-bit value}
\label{fig:A32BitExampleXDR}
\end{figure}

The number of bits inside of the \textit{bitfield} must be equal to or
less than the number of bits declared in the bitfield name.

You must declare them with the least significant bits first.
\textit{MyVariableName} in the example would be placed in
bit positions 0-3 and \textit{TheOtherFourBits} would occupy
bit position 4-7.

\textit{For signed values}, make sure that the value you pass
into the encoding call is correctly signed for the bit
width you specified in the XDR file.
If you specify a 4-bit value as signed
and pass it in as an 8-bit signed value, the sign bit will not
be encoded.

The \textit{bitfield} values can be signed or unsigned and
are XDR encoded and XDR decoded on their own.
Once encoded the \textit{bitfield} itself is sent as
opaque data.

In this example, the \textit{bitfield} is 64-bits and
contains multiple values that are not byte aligned:
\begin{figure}
\begin{verbatim}
bitfield ALargeOne:64 {
   int8_t:1 DeviceEnabled:1;     /* 'A' */
   int8_t:2 DeviceMode:2;        /* 'B' */
   int8_t:4 DeviceRotation:4;    /* 'C' */
   int8_t:4 DeviceStatus:4;      /* 'D' */
   int32_t:32 DevicePosition:32; /* 'E' */
};
\end{verbatim}
\caption{Packed 64-bit device example}
\label{fig:ALargeOneXDR}
\end{figure}
The upper 21 bits are undefined in this example,
and are set to zero (0) in this
implementation.
XDR and this implementation sends data in chunks of 32-bits.
The 'A', 'B', 'C', 'D', and 'E' comments in figure
\ref{fig:ALargeOneXDR} are represented as the bits in
the 64-bit figure \ref{fig:ALargeOneBitsXDR}.
\begin{itemize}
\item 'A' is one bit wide.
\item 'B' is two bits wide.
\item 'C' is four bits wide.
\item 'D' is four bits wide.
\item 'E' is 32 bits wide.
\end{itemize}
\begin{figure}
  \includesvg{64bitBitField.svg}
  \caption{64-bit 'ALargeOne' bitfield}
  \label{fig:ALargeOneBitsXDR}
\end{figure}

RPCGEN++ does not impose a size limit on the bit width.
This is an example of a 192-bits object.

\begin{figure}
\begin{verbatim}
bitfield A192BitOne:192 {
  ... define up to 192 bits ...
};
\end{verbatim}
\caption{Packed 192-bit device example}
\label{fig:A192BitXDR}
\end{figure}

\subsection{Added a \textit{class} object that extends \textit{struct}}
RPCGEN++ accepts the keyword \textit{class} to be substituted in
place of \textit{struct}.
When \textit{class} is used, the object may have member functions.
See \ref{ClassDefinition} for the full specification.
\begin{figure}
\begin{verbatim}
class OverTheWirePacket {
  Command    Cmd;
  uint8_t    Payload<>;

  /**
   * You can define a constructor
   *
   * @param WhatToDo the command to process.
   *
   * @param Data In this example, the contents of Data is determinable
   * from WhatToDo.
   */
  ObjectName(Command WhatToDo, uint8_t Data<>);

  /**  
   * When you define a constructor (like the previous line),
   * RPCGEN++ will only create the ones you specify. If you specify none,
   * it will generate a default constructor. So we also define a
   * a default constructor
   */
  ObjectName();

  /**
   * You can define a destructor
   */
  ~ObjectName();

  /**
   * You can define an unlimited number of member functions.
   * The 'const' signifies that the method GetTheAverageValues()
   * does not alter the contents of this object.
   *
   * @param Value1 Any kind or number of function parameters.
   *
   * @param Enabled Another parameter
   */
  float GetTheAverageValues(float Value1, bool Enabled) const;
};

\end{verbatim}
\caption{Example of new class keyword}
\label{fig:ClassIntroXDR}
\end{figure}

\subsection{The \textit{class} data type}\label{sec:ClassDefinition}
RPCGEN++ allows member functions to be placed inside
of class elements that are passed to the target
language as members of the object.

Member names are of the form:
\begin{figure}
\begin{verbatim}
;
; For output languages that do not have pointer or reference types,
; the pointer (*) or reference (&) is ignored.
;
; A method returns a TypeSpecifier type of data,
;
; Has a name (IDENTIFIER),
; and takes zero or more parameters.
;
; For languages that support 'const', it signifies that
; the named member is not allowed to modify the contents
; of the object, and has read-only access.
;
method: 'const'? TypeSpecifier '*'? IDENTIFIER '(' Param* (',' Param)* ')' 'const'? ';'

; An Param is of type Typespecifier, an optional pointer indicator '*',
; or an optional reference indicator '&', and an optional
; IDENTIFIER
.
Param : 'const'? TypeSpecifier ('*' | '&')? IDENTIFIER*

;
; A TypeSpecifier is any XDR data type
; OR any user defined data type.
;
TypeSpecifier : XdrDataType | UserDataType
\end{verbatim}
\caption{Class Method Definition}
\label{fig:ClassIntroXDR}
\end{figure}

\subsubsection{XDRDataType}
Where XdrDataType is defined
as \href {https://datatracker.ietf.org/doc/html/rfc4506#page-4}{XDR Data Types}

\subsubsection{IDENTIFIER}
An IDENTIFIER is a string starting with an upper or lower
case letter, followed by zero or more upper or lower case
letters or digits, or underline.

\begin{figure}
\begin{verbatim}
IDENTIFIER : [a-zA-Z] ([a-zA-Z0-9_])*
\end{verbatim}
\caption{IDENTIFIER Definition}
\label{fig:IDENTIFIERDefinitionXDR}
\end{figure}

\subsubsection{UserDataType}
A user data type, is any typedef, struct, enum, or class
name defined in the XDR file.
\begin{figure}
\begin{verbatim}
enum User1 { cat = 1, dog = 2, human = 3};

struct Car {
 uint16_t Year;
 string Make<>;
 string Model<>;
}

typedef struct Car Cars<>;
\end{verbatim}
\caption{UserDataType Definition}
\label{fig:UserDataTypeDefinitionXDR}
\end{figure}

In figure \ref{fig:UserDataTypeDefinitionXDR)}, User1, Car, and Cars
are examples of a UserDataType.

