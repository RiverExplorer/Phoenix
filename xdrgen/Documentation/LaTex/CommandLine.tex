\section{Command Line}
RPCGEN++ can currently generate:
\begin{itemize}
\item C++ code from XDR (.x) files.
\item C++ from ABNF (.abnf) files.
\item ABNF files from XDR files.
\end{itemize}

\subsection{Command Line Arguments}
Here is a list of RPCGEN++ command line options.
Not all of these options are available at any time.
See the sections on generating code.

\begin{itemize}

\item --abnf

  Optional: When the input files is an XDR file,
  this produces ABNF out files for the XDR definitions.
  
\item --client

  Optional: Create a client that understands the input grammar
  and can interact with servers.
  
\item -Dname[=value]

  Optional: May occur more than once.
  The input files are processed with the c-preprocessor 'cpp'
  command. All '-D' values are passed to the c-preprocessor command.
  
\item --headers

  Optional: Produce header files needed by the lang.
  As other output languages are added, if they have definition
  files, this will create their definition files.
  
\item --input input-file
  
  Required: The name of the input file to be processed.
  Files that end with '.x' will be processed as XDR files.
  Files that end with '.abnf'' will be processed as ABNF files.
  
\item --lang=CPP
  
  Optional: Optional. If not supplied, CPP is assumed.
  More output languages will be added later.
  Currently only 'CPP' is supported.
  
\item --outdir out-directory-name
  
  Optional: If not specified the files will be generated
  in a directory named 'Generated'.
  
\item --server

  Optional: Create a server that understands the input grammar
  and responds to clients.
  
\item --stubs

  Optional: Create the source code wrappers for users to
  use as their staring point to get, set, transmit, and
  receive these objects over the network.
  
\item --xdr

  Optional: Produce the source code. These are the source
  code files that match the --headers option.
  
\item --xds

  Optional: Produce an XML XSD file.
  
\end{itemize}

\subsection{Generating C++ code from XDR files}

\subsection{Generating C++ code from ABNF files}

\subsection{Generating ABNF from XDR files}

