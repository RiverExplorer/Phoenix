\section{Packet Details}
There are two basic kinds of packets in this implementation.
Internal packets \textit{IovCommand} and over the wire
packets \textit{Command}.
The \textit{Iov} object is separate because it is used
later in this specification for another purpose.

\subsection{I/O Vector Command}
\begin{figure}
  \centering
  \includesvg{Generated/IovPacket.svg}
  \caption{I/O Vector Command}
  \label{fig:IovPacket}
\end{figure}

An Iov packet is much like a system iovec that readv() might use.
Except is is a C++ class and methods and additional information.
The Iov object has a boolean that indicates if the segment
is memory mapped or allocated.

A memory mapped file does not have to be read into memory, while
the program can still has random access to the data in a virtual
memory segment.

Here is a stripped down version of \textit{Iov}.
The \textit{Iov} is essentially a \verb|std::deque<Blob*>| queue.
And the \textit{IovCommand} is an \textit{Iov} plus a command
and sequence.

\lstinputlisting[caption={Iov Source},label={code:Iov}]{Include/Iov.hpp}
% For some reason the above command removes the first character of the
% next word, so I added a '_' it will skip == hack fix.
_Pointers to \textit{Iov} objects, are what is stored in the
out queue \textit{OQ}
and in queue \textit{IQ} shown in figure \ref{fig:IOModel}

\subsection{Command Packet}
Here is a stripped down version of a single \textit{Command} and
a \textit{PacketBody} object that is sent over the wire.
The command packet is an over the wire packet, so it is defined in
the XDR (.x) format:

\lstinputlisting[caption={Iov Source},label={code:Iov}]{Include/PacketBody.x}
% For some reason the above command removes the first character of the
% next word, so I added a '_' it will skip == hack fix.
_Where \textit{SEQ\_t} is a unsigned 32-bit integer (\textit{uint32\_t}).

A sequence \textit{SEQ\_t} The endpoint that originates
the connection to the other endpoint only sends even
numbered sequence numbers in the commands.
The side that is connected to, only sends odd numbered
sequence numbers in the commands.
In this way an endpoint can quickly determine if the incoming packet
is a command, or a reply.

\textbf{NOTE: It is possible that specific implementations allow
  each endpoint to act as a client and server.
  For the purposes of this specification, the client is
  the one that is originating actions and expects
  results or replies from the other endpoint}

The side that initiates the connection only sends
even numbered sequence numbers in the commands.
Even when acting in a dual role.

The side that gets connected to, only sends
odd numbered sequence numbers in the commands.
Even when acting in a dual role.
