\subsection{File and Folder Commands}
A folder is not necessarily a directory on the server.
It could be a logical folder in another object.
It might be a specific email message in a email folder container.
It could be a specific file in a zip or tar file.
It could be an entry in a database.

Within the Phoenix protocol a folder is simply a container
of any kind that can perform one or more of the folder
and file commands Phoenix supports.
Not all folders may support all commands.

Examples: Some may be read only.
Others might not allow them to be renamed.

\subsubsection{FILE\_CREATE}
Create a new file.
What is creates is a function of the task the
server is being asked to do:

Examples of what it could be doing:
\begin{enumerate}
\item When it is processing email messages, it could be creating
  a new email message.
  When sent to an outbound email server, it sends email.
  When sent to a user email server, it could be saving a draft.
\item When it is processing calendar messages, it could be creating
  a new calendar event, todo, or other calendar object.
\item It could be creating a new file on a disk.
\end{enumerate}

See the specific task specifications.

A FILE\_CREATE command includes the ID of the folder
to create.
And if applicable, a path to the new file.
The path separation character is the slash (/) 0x2f UTF-8 character.
\subsubsection{FILE\_COPY}

\subsubsection{FILE\_DELETE}
\subsubsection{FILE\_RENAME}
\subsubsection{FILE\_METADATA}
\subsubsection{FILE\_MOVE}
\subsubsection{FILE\_SHARE}
\subsubsection{FILE\_GET}
\subsubsection{FILE\_MODIFY}
\subsubsection{FOLDER\_CAPABILITY}
\subsubsection{FOLDER\_CREATE}
Create a new folder.
What it creates is a function of the task the
server is being asked to do.

Examples of what it could be doing:
\begin{enumerate}
\item When it is processing email messages, it could be creating
a new email folder.
\item When it is processing calendar messages, it could be creating
  a new calendar.
\item Performing actual disk folder (directory) commands.
\end{enumerate}

See the specific task specifications.

A FILE\_CREATE command includes the name of the folder
to create.
And if applicable, a path.
The path separation character is the slash (/) 0x2f UTF-8 character.
\subsubsection{FOLDER\_COPY}
\subsubsection{FOLDER\_DELETE}
\subsubsection{FOLDER\_RENAME}
\subsubsection{FOLDER\_METADATA}
\subsubsection{FOLDER\_MOVE}
\subsubsection{FOLDER\_OPEN}
\subsubsection{FOLDER\_SHARE}
\subsubsection{FOLDER\_LIST}
