\section{Commands Overview}
Commands are all \ref{code:PacketBody}{ PacketBody} objects.

The command descriptions are broken down into related sections in order
to simplify the documentation.

The client connects to the server, and then the
server waits for a AUTHENTICATION command or a PRE
authentication CAPABILITY command from the client.

There are PRE and POST CAPABILITY commands.
The PRE CAPABILITY command have to do with establishing
an authenticated user or client connection.
The POST authentication CAPABILITY command informs
the other endpoint what it can do for the other endpoint
on behalf of this now authenticated user or client.

One or more commands are sent in each PacketBody
sent between the client and server.

There is no guarantee that any two clients and server
can authenticate or process any data.
The Phoenix protocol is an application level MIME transport
protocol and the two endpoints might not implement the
same MIME types.
For example one might do only email, and the other might do
only RSS feeds.
Other servers and clients may be capable of multiple tasks.

Commands can be documented commands or vendor commands.
A vendor may publicly document its extensions, or they
may elect to keep them proprietary and unavailable to other vendors.
Vendor commands are allowed because the purpose of this protocol
is to facilitate the transfer of MIME objects.
It is not to guarantee every implementation can transfer
any data to or through any other implementation.

\subsection{Authentication Commands}
AUTHENTICATION commands is the Authentication, authorization,
accounting (AAA) process between a client and server.

More than one authentication mechanisms may be required
by a server.
For example a server may require AUTHCERT\_TLS and AUTHCERT\_USER
or perhaps AUTHCERT\_TLS and AUTHMD5.

\subsubsection{New Client and Server Relationship}
When a client has not had a successful authentication to a
specific server, it sends its first packet as a PRE authentication
CAPABILITY packet informing the server of the authentication mechanisms
it supports.
The list is prioritized.
With the most preferred listed first.
And the least preferred ones later in the list.

Once the server gets this list, it picks the highest
priority one it supports then then sends the client
the servers PRE authentication CAPABILITY command listing only
that authentication mechanism.
The server may include any other PRE authentication CAPABILITY
items to the client it may have in that command.

If no common authentication mechanism are satisfactory,
the server sends the client a BYE
command and closes the connection.

If the client gets a BYE command, it then knows that it can
not communicate with this server and also closes its connection.

If the client gets the servers PRE authentication CAPABILITY
command, it attempts the agreed on authentication process
using any possible additional CAPABILITY description
restrictions or conditions provided in the server
PRE authentication CAPABILITY contents.

When the agreed on authentication mechanism works,
the server sends its POST authentication CAPABILITY
list to the client.

When the agreed on authentication mechanism fails,
the server a PRE authentication CAPABILITY command
to the client.
It can be the same as sent before, or modified.

A server may at any time block or temporally block an IP address
from attempting to start authentication again or too
quickly after a failure.

\subsubsection{Existing Client and Server Relationship}
When a client has had a relationship to a specific server
using a specific authentication mechanism it may bypass
the PRE authentication CAPABILITY communications and send an
AUTHENTICATION command when first connecting in order to reduce
the round trips needed to start processing other commands.
This is done by sending the AUTHENTICATION mechanism that had
previously worked with this server as the first command
in a PacketBody that also includes the clients PRE authentication
CAPABILITY command.
The first command is the AUTHENTICATION and the second included
command is the clients PRE authentication CAPABILITY command.
When this authentication succeeds, the server sends its POST
authentication CAPABILITY back to the client as the indicator
that it worked and the client is authenticated.

If this initial authentication fails, client is not yet authenticated.
And the server having already received the clients PRE authentication
CAPABILITY command, picks the highest priority authentication
mechanism they share and sends the client the servers PRE authentication
command containing any other CAPABILITY items it may need to share.

Then the authentication continues until success or failure.

When a server, after examining the received client PRE authentication
CAPABILITY command determines that they do not share any
common authentication, the server sends the client a BYE
command and closes the connection.
The BYE command at this step informs the client that they
share no common authentication mechanisms and also
closes the connection.

\subsubsection{AUTHANONYMOUS}
An anonymous authentication would be allowed on
servers that do not require a user to login to the system.
An example could be a \textit{Really Simple Syndication} (RSS)
feed that provides the latest product release notes available to anyone.

Server administration need to ensure that absolutely anyone
with network access has permission to connect to an AUTHANONYMOUS
server.

\subsubsection{AUTHCERT\_TLS}
TLS authentication is done when the client connects to the
server.

They each validate the TLS certificate and certificate chain
of the other.

If they validate, the client is authenticated to the server.

If they do not validate, the client may choose another authentication
mechanism or disconnect from the server.

A server may disconnect from the client if AUTHCERT\_TLS is required.

\subsubsection{AUTHCERT\_USER}
AUTHCERT\_USER is a secondary certificate that is used
after the initial TLS connection.

For example, if the user is travailing, they may have
a portable certificate tool or software that is used
to ensure that unknown or unfriendly hardware can not
view the communications.

The steps are:
\begin{enumerate}
\item Establish the TLS connection.
\item Using the user certificate, encrypt all communication
  from this point forward.
  Doubling the encryption process.
\end{enumerate}

A client that implements AUTHCERT\_USER would need a way
for a user to connect a device to the client in order
to allow the data flow to go through the user supplied hardware.
An example could be a specialty USB device that encrypts the data
between the network and the client.

\subsubsection{AUTHMD5}
MD5 authentication is used in many protocols.
The client, over a TLS connection sends the account name
and the MD5 hashed password to the server.

\subsubsection{BYE}
When the client or server receives a BYE command,
it has been informed that the other endpoint has closed
or is closing the connection and that no further communications
can take place on this connection.

\subsection{Capability Commands}
Capability commands are not a negotiation.
Each endpoint informs the other endpoint of the type
of transmissions it can process.

There is no guarantee that any two clients and server
can process any data.
The Phoenix protocol is an application level MIME transport
protocol and the two endpoints might not implement the
same MIME types.
For example one might do only email, and the other might do
only calendar feeds.
Other servers and clients may be capable of multiple tasks.

\subsubsection{CAPABILITY\_PRE}
A PRE authentication CAPABILITY is used before
the endpoints are authenticated and contain minimal
information.

At a minimum a PRE authentication capability includes
a prioritized list of authentication mechanisms it supports.
It also includes a VENDOR\_ID which is used to let each
endpoint know if they can transfer or perform vendor specific
data or communications.

\subsubsection{CAPABILITY\_POST}
A POST authentication CAPABILITY is sent after a successful
authentication informing the other endpoint of the functions
that it can perform.

A server may include the VENDOR\_ID value again, if after
authentication the server determines the client an perform
actions that are unique between the client and server.

A vendor may publicly document its extensions, or they
may elect to keep them proprietary and unavailable to other vendors.
\subsubsection{VENDOR\_ID}

\subsection{Control Commands}
These are housekeeping command for the Phoenix protocol.

\subsubsection{NOT\_SUPPORTED}
A NOT\_SUPPORTED command is sent as the reply to a command
not supported by the receiving endpoint.
The sequence number of the received command is sent in
the NOT\_SUPPORTED command to indicate which command
is not supported.

\subsubsection{PING}

The ping command is only sent when the client implementation has
determined it has waited too long for a command reply.
The ping command is only initiated from the client.
It is not valid for the server to send a ping command to a client.

A ping command must not be sent before a successful authentication.

The ping command MUST NOT be the first command sent to the server.
It should only be sent when the client implementation determines
it has waited too long for a reply.

If the server supports the ping command, then a PING capability is
sent in the CAPABILITY\_POST command.

Sometimes servers are unavailable and can go down.
A server could be down for maintenance, or in a shutdown mode.
It might limit the number of simultaneous connections.
It might be very busy.
The packets might not be making it to the server because of network
issues.

When a ping command is received by the server:
\begin{enumerate}
\item When the server did not send PING capability in the post
authentication capability list to the client.
The server ignores the PING command.
\item When the connection is not authenticated,
  The server ignores the PING command.
\item When the client is authenticated, and when the server is
  available for processing commands.
  Then the server replies with a ping reply with the same sequence number.
\end{enumerate}

If the server is alive and not available, the server will reply with
a NotSupported command, with its sequence number set to the sequence
number in the ping command.

If a connected and authenticated client has been waiting for a reply
or for some other reason needs to determine if the server is still
available.
It can send a ping command. If the server is still available, it
sends a ping reply.
If it is no longer available for any reason, it sends
a NotSupported reply.

A client MUST NOT send a ping command if it is waiting the results
of a previously sent ping command.
If the server is ignoring PING commands, a reply will never happen.

A client MUST NOT send a ping command more frequently than 90\% of
the SERVER\_TIMEOUT value that the server sent in the
CAPABILITY\_POST command.

Servers must give priority to ping commands.
If possible, reply as soon as it receives the command.

With servers that support PING, clients MUST NOT send any other
command while waiting for the PING reply.

The server MAY consider too many ping commands as a
malfunctioning or malicious client and terminate the connection.

Servers that are not threaded or can not reply to simultaneous
or overlapping commands, MUST NOT include PING in their post
authentication capability command.

\subsubsection{RESERVED\_CMD}
The RESERVED\_CMD is a placeholder for future extensions.
Its value is 0xffffffff.

\subsection{File and Folder Commands}
A folder is not necessarily a directory on the server.
It could be a logical folder in another object.
It might be a specific email message in a email folder container.
It could be a specific file in a zip or tar file.
It could be an entry in a database.

Within the Phoenix protocol a folder is simply a container
of any kind that can perform one or more of the folder
and file commands Phoenix supports.
Not all folders may support all commands.

Examples: Some may be read only.
Others might not allow them to be renamed.

\subsubsection{FILE\_CREATE}
Create a new file.
What is creates is a function of the task the
server is being asked to do:

Examples of what it could be doing:
\begin{enumerate}
\item When it is processing email messages, it could be creating
  a new email message.
  When sent to an outbound email server, it sends email.
  When sent to a user email server, it could be saving a draft.
\item When it is processing calendar messages, it could be creating
  a new calendar event, todo, or other calendar object.
\item It could be creating a new file on a disk.
\end{enumerate}

See the specific task specifications.

A FILE\_CREATE command includes the ID of the folder
to create.
And if applicable, a path to the new file.
The path separation character is the slash (/) 0x2f UTF-8 character.
\subsubsection{FILE\_COPY}

\subsubsection{FILE\_DELETE}
\subsubsection{FILE\_RENAME}
\subsubsection{FILE\_METADATA}
\subsubsection{FILE\_MOVE}
\subsubsection{FILE\_SHARE}
\subsubsection{FILE\_GET}
\subsubsection{FILE\_MODIFY}
\subsubsection{FOLDER\_CAPABILITY}
\subsubsection{FOLDER\_CREATE}
Create a new folder.
What it creates is a function of the task the
server is being asked to do.

Examples of what it could be doing:
\begin{enumerate}
\item When it is processing email messages, it could be creating
a new email folder.
\item When it is processing calendar messages, it could be creating
  a new calendar.
\item Performing actual disk folder (directory) commands.
\end{enumerate}

See the specific task specifications.

A FILE\_CREATE command includes the name of the folder
to create.
And if applicable, a path.
The path separation character is the slash (/) 0x2f UTF-8 character.
\subsubsection{FOLDER\_COPY}
\subsubsection{FOLDER\_DELETE}
\subsubsection{FOLDER\_RENAME}
\subsubsection{FOLDER\_METADATA}
\subsubsection{FOLDER\_MOVE}
\subsubsection{FOLDER\_OPEN}
\subsubsection{FOLDER\_SHARE}
\subsubsection{FOLDER\_LIST}

\subsection{Administrative Commands}
\subsubsection{SERVER\_CONFIGURE}
\subsubsection{SERVER\_KICK\_USER}
\subsubsection{SERVER\_LOGS}
\subsubsection{SERVER\_MANAGE\_BANS}
\subsubsection{SERVER\_SHUTDOWN}
\subsubsection{SERVER\_VIEW\_STATS}
\subsubsection{USER\_CREATE}
\subsubsection{USER\_DELETE}
\subsubsection{USER\_LIST}
\subsubsection{USER\_PERMISSIONS}
\subsubsection{USER\_RENAME}


