\section{Introduction}
The client and the server send data to the network, and get
data from the network.
This \mbox{input/output} model (IO Model) works with
both the Phoenix client and Phoenix server.

\textbf{NOTE: It is possible that specific implementations allow
  each endpoint to act as a client and server.
  For the purposes of this specification, the client is
  the one that is originating actions and expects
  results or replies from the other endpoint}

This model support both threaded and non threaded
implementations.
For non-threaded systems the data is processed serially
or however the implementation handles both incoming
and outgoing data.
The data flow is the same for both threaded and non threaded
implementations.

\begin{figure}
  \centering
  \caption{I/O Model}
  \label{fig:IOModel}
  \includesvg{Generated/IO-00.svg}
\end{figure}

\begin{description}
\item[(A)Main Thread]\label{item:MainThread}
	The main thread processes data and formats it into
  IOVectorPacket objects that contains the command
  and references to the data that will be sent.
  Then stores the IOVectorPackt (figure: \ref{fig:IovPacket})
  into queue \textit{OQ}.
  The client \textit{Main Thread} is described in section \ref{sec:ClientMainThread}
  And the server \textit{Main Thread} is described in section \ref{sec:ServerMainThread}

\item[(B) Output Thread]
  This thread waits for the output queue \textit{OQ} to have
  content.
  It checks to make sure that the network is writable and not write blocked
  or otherwise busy.
  When it can proceed, it takes an entry out of the queue,
  and (C) XDR encodes it.
  The XDR stream is opened to the TLS encryption.
  The \textit{Output Thread} is described in section \ref{sec:OutputThread}
  
\item[(C) XDR Encoding]\label{item:XdrEncodingThread}
	This thread, takes one packet at at time
	out of the input queue,	converts it from the native computer
  to network compatible XDR format.
  It reads the IOVectorPacket data to an XDR stream and that streams is sent directly
  to the \textit{(D) TLS Encryption} code.
  \textit{XDR Encoding} is described in section \ref{sec:XdrEncode}

\item[(D) TLS Encryption]\label{item:TlsEncryption}
  This code takes the XDR encoded data from its input stream
  and encrypts it to the network.
  Which encrypts it according to the certificates supplied by the configuration.
  and sends it out to the network.
  \textit{TLS Encryption} is described in section \ref{sec:TlsIOOut}

\item[(E) Input Thread]\label{item:InputThread}
  This thread waits for any data coming from the network
  and feeds it into the \textit{TLS Decryption} code.
    
\item[(F) TLS Decryption]\label{item:TlsDecryptionThread}
  This code decrypts it on the fly, and streams it to \textit{(G) XDR Decoding} code.
  \textit{TLS Decryption} is described in section \ref{sec:TlsIOIn}

\item[(G) XDR DECODING]\label{item:XdrDecoding}
  This takes data from the \textit{TLS Decryption} code and decodes
  it back to native computer data format.
  \textit{XDR Decoding} is described in section \ref{sec:XdrDecode}
  
\item[(H) Dispatch Thread]\label{DispatchThread}
	This is the dispatch thread.
  It takes the decoded packet and dispatches it to the code that can handle the data.
  The \textit{Dispatch Thread} is described in section \ref{sec:DispatchThread}

\item[(OQ) Output Queue]
  This is a C++ \textit{std::deque} of I/O Vector Packets
  shown in figure \ref{fig:IovPacket}.
  
\item[(IQ) Input Queue]
  This is a C++ \textit{std::deque} of I/O Vector Packets
  shown in figure 1\ref{fig:IovPacket}.
  
\end{description}


This model is the core of both the client and server.
